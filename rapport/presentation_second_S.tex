\documentclass{beamer}
\usepackage[utf8]{inputenc}
\usepackage[frenchb]{babel}
\usepackage[T1]{fontenc}
\usepackage{graphics}
\usepackage{framed}
\usepackage{graphicx}
\usepackage{subcaption}
\usepackage{grffile}
\usepackage{longtable}
\usepackage{wrapfig}
\usepackage{rotating}
\usepackage[normalem]{ulem}
\usepackage{amsmath}
\usepackage{textcomp}
\usepackage{amssymb}
\usepackage{capt-of}
\usetheme{Madrid}

\title[Support pour Verificarlo]{Support de MPI et de la vectorisation dans Verificarlo}
\subtitle{Master Calcul Haute Performance et Simulation}

\author[Hery, Nicolas, Ali]{Hery ANDRIANANTENAINA \\ Nicolas BOUTON \\ Ali LAKBAL}

\institute[]{\textbf{Encadrant:} Eric PETIT}

\date{Année 2020-2021}

\begin{document}

\maketitle

\begin{frame}{Support de la parallélisation}
    \begin{block}{Introduction}
      \includegraphics[width=0.6\linewidth]{index.jpeg}
      source:fr.wikipedia.org
    \end{block}

\end{frame}

\begin{frame}{Différentes types de NAS Benchmarks Parallèle}

    \begin{block}{NPB1}
      
        \begin{itemize}
            \item Introduction des fonctions qui va facilité les parallélismes
            \item Vérification de la cohérence des résultats et de la performance
            \item Adaptation avec les systèmes à plusieurs coeurs.
        \end{itemize}
    \end{block}
    
    \begin{block}{NPB2}
         \begin{itemize}
            \item Modification des règles sur les analyses comparative
            \item Accès aux codes sources pour le grand public
            \item Version parallélisé avec MPI
        \end{itemize}
    \end{block}
    
    \begin{block}{NPB3}
      \begin{itemize}
          \item Version parallélisé avec openmp
          \item Version hybride ou "Multi-zone" 
      \end{itemize}
    \end{block}

\end{frame}
 
\begin{frame}{ Différentes types de benchmark}
    \begin{block}{Types}
      \begin{itemize}
        \item IS : "Integer Sort"
        \item EP : " Embarrassingly Parallel"
        \item CG : "Conjugate Gradient" 
        \item MG : "Multi-Grid"
        \item FT : "Fourier Transform"
        \item BT: "Block Tri-diagonal solver"
        \item SP: "Scalar Penta-diagonal solver"
        \item LU: "Lower-Upper Gauss-Seidel solver"
    \end{itemize}
    \end{block}
        
    \begin{block}{Classe}
      \begin{itemize}
        \item Classe S 
        \item  Classes A , B , C 
        \item Classes D , E , F
      \end{itemize}
    \end{block}
    
\end{frame}

\begin{frame}{ Résultat et discussion}
    \begin{block}{Compilation}
      CC=OMPI\_FC=verificarlo-f  mpif90
    \end{block}
    
    \begin{block}{Problème}
      Dimension sept pour les tableaux
    \end{block}
    
    \begin{block}{Solution}
      Ré-compilation
      \begin{itemize}
          \item CC=clang
          \item CXX=clang++
          \item FC=flang
      \end{itemize}
    \end{block}
\end{frame}

\begin{frame}{Résultat et discussion}
    \begin{block}{Test}
      \includegraphics[width=0.6\linewidth]{../ressources/btcompleted.png}
    \end{block}
\end{frame}

\begin{frame}{Résultat et discussion}
    \begin{block}{Résultat vectorisation NPB-MPI FORTRAN}
      \includegraphics[width=0.6\linewidth]{../ressources/vect1.png}
    \end{block}
    
    \begin{block}{Résultat vectorisation NPB-OPENMP C}
      \includegraphics[width=0.6\linewidth]{../ressources/vcopenmp.png}
    \end{block}
\end{frame}

\begin{frame}{ Conclusion}
    \begin{block}{Test super-calculateur}
      \begin{itemize}
          \item Évaluation de la vectorisation sur le jeu d'instruction AVX512
          \item Faire des tests sur les problèmes de tailles standards ou les gros problèmes
      \end{itemize}
    \end{block}
    
    \begin{block}{Conclusion générale}
    
        \begin{itemize}
            \item  Le module de Calcul Numérique sur la précision numérique qui est un des sujets principaux de verificarlo.
            \item Le module d’Architecture Parallèle et de Technique d’Optimisation Parallèle pour les mesures de performances et les benchmarks.
            \item  les modules traitant la parallélisation avec MPI et OpenMP comme Algorithme de Programmation Parallèle 
        \end{itemize}
     
    \end{block}
\end{frame}
\end{document}
