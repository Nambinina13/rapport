\documentclass{beamer}
\usepackage[utf8]{inputenc}
\usepackage[frenchb]{babel}
\usepackage[T1]{fontenc}
\usepackage{graphics}
\usepackage{framed}
\usepackage{graphicx}
\usepackage{subcaption}
\usepackage{grffile}
\usepackage{longtable}
\usepackage{wrapfig}
\usepackage{rotating}
\usepackage[normalem]{ulem}
\usepackage{amsmath}
\usepackage{textcomp}
\usepackage{amssymb}
\usepackage{capt-of}
\usetheme{Madrid}

\title[Support pour Verificarlo]{Support de MPI et de la vectorisation dans Verificarlo}
\subtitle{Master Calcul Haute Performance et Simulation}

\author[Hery, Nicolas, Ali]{Hery ANDRIANANTENAINA \\ Nicolas BOUTON \\ Ali LAKBAL}

\institute[]{\textbf{Encadrant:} Eric PETIT}

\date{Année 2020-2021}

\begin{document}

\maketitle

% Nicolas
\begin{frame}{Verificarlo}

  \begin{block}{Compilateur s'appuyant sur}
    \begin{itemize}
    \item clang
    \item llvm
    \end{itemize}
  \end{block}

  \begin{block}{But}
    Intercepter les opérations flottantes afin de les analyser et les débogger
  \end{block}
  
\end{frame}

\begin{frame}{Rappel du backend VPREC}

  \begin{figure}
    \centering
    \includegraphics[width=200px]{../ressources/ieee754_floating-point_single_precision.jpg}
    \caption{\label{fig:ieee754_single_precision}Représentation d'un nombre flottant simple précision}
  \end{figure}

  \begin{block}{Cas spéciaux}
    \begin{itemize}
    \item NaN (Not a Number - Pas un Nombre)
    \item nombres infinis
    \item nombres dénormaux
    \end{itemize}
  \end{block}

\end{frame}

\begin{frame}{Changements aux niveaux du backend VPREC}

  \begin{figure}
    \centering
    \includegraphics[width=300px]{../ressources/op_normal}
    \caption{\label{fig:ieee_simple_precision}Addition de 2 vecteurs de 8
      flottants simple précision générant un vecteur contenant que des nombres normaux}
  \end{figure}

\end{frame}

\begin{frame}{Changements aux niveaux du backend VPREC}

  \begin{figure}
    \centering
    \includegraphics[width=300px]{../ressources/op_infini}
    \caption{\label{fig:ieee_simple_precision}Addition de 2 vecteurs de 8
      flottants simple précision générant un vecteur contenant un élément infini}
  \end{figure}

\end{frame}

% Ali
\begin{frame}{Benchmark}
    \begin{block}{But}
      \begin{itemize}
          \item Tester les performances de l'implementation vectorielle par rapport à la version scalaire. 
          
      \end{itemize}
    \end{block}
  \begin{block}{Utilisation}
 
        \begin{itemize}
            \item Les backends IEEE et VPREC
            \item Les jeux d'instructions SSE et AVX
            \item Le type simple précision.
        \end{itemize}
  \end{block}
\end{frame}



\begin{frame}{Benchmark}
    \begin{block}{Micro-benchmark}
      \begin{itemize}
          \item Les boucles qui font les calculs que l'on mesure 
          
      \end{itemize}
    \end{block}
  \begin{block}{Metriques prises en compte}
 
        \begin{itemize}

            
            \item Le temps.
            \item L'écart type.
            \item L'accelération.
           
        \end{itemize}
  \end{block}
\end{frame}

\begin{frame}{Tests et résultats}

  \begin{block}{Tests}
 
        \begin{itemize}

            \item Plusieurs tentatives d'execution.
            \item Machine virtuelle : accepte AVX
            \item Linux natif : n'accepte pas AVX.
           
           
        \end{itemize}
  \end{block}
\end{frame}



\begin{frame}{Tests et résultats}
    \begin{block}{VPREC}
    \begin{itemize}
        \item écart type sur la VM 
    \end{itemize}
 \centering\includegraphics[width=160px]{../ressources/vm_vprec_normal_stddev.png}
  \centering\includegraphics[width=160px]{../ressources/vm_vprec_denormal_stddev.png}
    \end{block}
\end{frame}

\begin{frame}{Tests et résultats}
    \begin{block}{VPREC}
    \begin{itemize}
        \item écart type sur Linux natif
    \end{itemize}
 \centering\includegraphics[width=160px]{../ressources/laptop_vprec_normal_stddev.png}
 
  \centering\includegraphics[width=160px]{../ressources/laptop_vprec_denormal_stddev.png}
    \end{block}
\end{frame}



\begin{frame}{Tests et résultats}
    \begin{block}{VPREC}
    \begin{itemize}
        \item Résultat d'accélération normaux/dénormaux
    \end{itemize}
 \centering\includegraphics[width=300px]{../ressources/vm_vprec.png}
    \end{block}
\end{frame}

\begin{frame}{Tests et résultats}
    \begin{block}{IEEE}
    \begin{itemize}
        \item écart type
    \end{itemize}
 \centering\includegraphics[width=300px]{../ressources/vm_ieee_stddev.png}
    \end{block}

\end{frame}

\begin{frame}{Tests et résultats}
    \begin{block}{IEEE}
    \begin{itemize}
        \item accélération
    \end{itemize}
 \centering\includegraphics[width=300px]{../ressources/vm_ieee.png}
    \end{block}
    
\end{frame}
% Hery
\begin{frame}{Support de la parallélisation}

  \begin{block}{Introduction}
    \includegraphics[width=0.6\linewidth]{../ressources/index.jpeg}
    source:fr.wikipedia.org
  \end{block}

\end{frame}

\begin{frame}{Différentes types de NAS Benchmarks Parallèle}

  \begin{block}{NPB1}
    \begin{itemize}
    \item Fonction
    \item Résultats et performance
    \item Systèmes à plusieurs coeurs.
    \end{itemize}
  \end{block}
  
  \begin{block}{NPB2}
    \begin{itemize}
    \item Analyses comparative
    \item Codes sources
    \item Parallélisme  MPI
    \end{itemize}
  \end{block}
  
  \begin{block}{NPB3}
    \begin{itemize}
    \item Parallélisme OPENMP
    \item Version hybride ou "Multi-zone" 
    \end{itemize}
  \end{block}

\end{frame}

\begin{frame}{ Différentes types de benchmark}

  \begin{block}{Types}
    \begin{itemize}
    \item CG : "Conjugate Gradient" 
    \item BT: "Block Tri-diagonal solver"
    \item LU: "Lower-Upper Gauss-Seidel solver"
    \end{itemize}
  \end{block}
  
  \begin{block}{Classe}
    \begin{itemize}
    \item Classe S 
    \item  Classes A , B , C 
    \item Classes D , E , F
    \end{itemize}
  \end{block}
  
\end{frame}

\begin{frame}{ Résultat et discussion}

  \begin{block}{Compilation}
    CC=OMPI\_FC=verificarlo-f  mpif90
  \end{block}
  
  \begin{block}{Problème}
    Dimension sept pour les tableaux
  \end{block}
  
  \begin{block}{Solution}
    Ré-compilation
    \begin{itemize}
    \item CC=clang
    \item CXX=clang++
    \item FC=flang
    \end{itemize}
  \end{block}

\end{frame}

\begin{frame}{Résultat et discussion}

  \begin{block}{Test}
    \includegraphics[width=0.6\linewidth]{../ressources/btcompleted.png}
  \end{block}

\end{frame}

\begin{frame}{Résultat et discussion}

  \begin{block}{Résultat vectorisation NPB-MPI FORTRAN}
    \includegraphics[width=0.6\linewidth]{../ressources/vect1.png}
  \end{block}
  
  \begin{block}{Résultat vectorisation NPB-OPENMP C}
    \includegraphics[width=0.6\linewidth]{../ressources/vcopenmp.png}
  \end{block}

\end{frame}

\begin{frame}{ Conclusion}

  \begin{block}{Test super-calculateur}
    \begin{itemize}
    \item Évaluation de la vectorisation sur le jeu d'instruction AVX512
    \item Faire des tests sur les problèmes de tailles standards ou les gros problèmes
    \end{itemize}
  \end{block}
  
\end{frame}

\end{document}
